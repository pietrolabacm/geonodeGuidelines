\documentclass[12pt]{article}

\usepackage{sbc-template}
\usepackage{graphicx,url}
\usepackage[utf8]{inputenc}
\usepackage{fancyhdr}
\usepackage[brazil]{babel}
\usepackage{hyperref}
\usepackage{xcolor}
%\usepackage[latin1]{inputenc}  

\pagenumbering{gobble}                          % sets page numbering off
\pagestyle{fancy}\setlength\headheight{100pt}   % set the pagestyle to fancy to include headers and footers
\pagestyle{fancy}\setlength\textheight{650pt}
\pagestyle{fancy}\setlength\footskip{20pt}
\fancyhf{}                                      % clear the header and footer
\renewcommand{\headrulewidth}{0pt}              % no horizontale ruler under header
\fancyhead[L]{\includegraphics[width=3.8cm]{img/LogoINCT.jpg}}
\fancyhead[R]{\includegraphics[width=3.8cm]{img/LogoOdissea.jpg}}

\begin{document} 

%\title{Guia para o carregamento de dados no Geonode Odisseia}
%\date{}
%\maketitle
%\thispagestyle{fancy}

\begin{center}
  \vspace{12pt}
  \Large\textbf{Guia para construção de páginas utilizando o app Home do Geonode}
\end{center}

\section{Introdução}

Este documento define as funcionalidades do aplicativo Home para o Geonode e
traz instruções para seu uso. O aplicativo foi criado com o objetivo de atender
as demandas de customização do Geonode do INCT Odisseia de forma que a
customização seja flexível, manejável e aceite futuras atualizações do Geonode.

\section{Subsites} \label{sec:firstpage}

O aplicativo Home estende as funcionalidades do aplicativo \textcolor{blue}{
\href{https://github.com/geosolutions-it/geonode-subsites}{geonode-subsites}}
que por sua vez é uma contribuição da comunidade ao Geonode. Este modulo
permite a subdivisão de um catálogo do Geonode em subcatálogos sem a
necessidade de se criar diferentes ambientes e subdividir os dados manualmente.

\section{O aplicativo Home}

O aplicativo Home permite que o administrador defina nome, descrição e imagem
de visualização para cada subsite, para que então sejam expostos em uma
homepage personalizada. Além disso, é possível adicionar categorias que, assim
como os subsites, possuem nome, descrição e imagem de visualização para compor
uma página de categorias para cada subsite.

O módulo foi construído de forma que o administrador pode modificar os filtros
e demais características dos subsites sem modificar o código-fonte do Geonode e
de seus aplicativos. Apenas utilizando a interface gráfica do administrador do
Django. Isso também permite que o módulo seja reutilizado em futuras
atualizações do Geonode, a princípio sem nenhuma alteração.

\subsection{Tipos de dados}

O aplicativo cria duas tabelas no banco de dados que hospeda as informações
utilizadas pelo Geonode, uma tabela para os subsites e uma para as categorias.
Cada subsite gera uma entrada em sua tabela contendo 14 colunas, dentre elas 11
colunas herdadas do aplicativo \textcolor{blue}{
\href{https://github.com/geosolutions-it/geonode-subsites}{geonode-subsites}} e
3 colunas exclusivas do aplicativo Home, estas são: "display\_ name",
"display\_ description" e "display\_ image". 

\begin{description}
  \setlength\itemsep{1em}
  \item[display\_name]
    - Um objeto da classe 
      \textcolor{blue}
      {\href{https://docs.djangoproject.com/en/5.0/ref/models/fields/#django.db.models.CharField}
      {CharField}} 
    do Django que gera uma coluna do tipo VARCHAR na tabela do
    banco de dados e possui um limite de 50 caracteres. Este nome será
    exposto como o título do subsite na homepage do Geonode.
    
  \item[display\_description]
  - Um objeto da classe 
      \textcolor{blue}
      {\href{https://docs.djangoproject.com/en/5.0/ref/models/fields/#django.db.models.CharField}
      {CharField}} 
    do Django que gera uma coluna do tipo VARCHAR com um limite de
    50 caracteres. Este nome será exposto como a descrição do subsite na
    homepage do Geonode.

  \item[display\_image]
  - Um objeto da classe
      \textcolor{blue}
      {\href{https://docs.djangoproject.com/en/5.0/ref/models/fields/#django.db.models.ImageField}
      {ImageField}} 
    do Django que gera uma coluna do tipo VARCHAR com um limite
    de 100 caracteres contendo o caminho para a imagem dentro do volume de
    armazenamento do Django. Esta imagem será exposta na homepage como a imagem
    de referência do subsite. 
\end{description}

Cada categoria cria uma entrada em sua tabela com 6 colunas: "display\_name",
"display\_description", "display\_image", "fk", "keyword" e
"keywords\_filter\_url". As colunas de display possuem o mesmo comportamento da
tabela de subsites e tem como propósito construir a página de categorias de
cada subsite. Já as demais colunas relacionam a categoria a um subsite
específico e definem as keywords utilizadas para filtrar o dado.

\begin{description}
  \setlength\itemsep{1em}

  \item[fk]
  - Um objeto da classe
  \textcolor{blue}
    {\href{https://docs.djangoproject.com/en/5.0/ref/models/fields/#django.db.models.ForeignKey}
    {ForeignKey}}
  do Django que gera uma coluna de relacionamento 1:N correlacionando a categoria com o índice de seu respectivo subsite. 

  \item[keyword]
  - Um objeto da classe
    \textcolor{blue}
    {\href{https://docs.djangoproject.com/en/5.0/ref/models/fields/#django.db.models.ManyToManyField}
    {ManyToManyField}} 
  do Django que gera uma coluna de relacionamento N:N que
  correlaciona por meio de uma tabela intermediária uma categoria com um
  grupo de keywords que podem ser utilizados como filtro. Esses keywords são do modelo de 
    \textcolor{blue}
    {\href{https://docs.geonode.org/en/master/admin/admin_panel/index.html#hierarchical-keywords}
    {HierarchicalKeywords}}
  do Geonode.

  \item[keywords\_filter\_url]
  - Um objeto da classe 
    \textcolor{blue}
    {\href{https://docs.djangoproject.com/en/5.0/ref/models/fields/#django.db.models.CharField}
    {CharField}} 
  do Django que gera uma coluna do tipo VARCHAR com um limite de
  200 caracteres. Este campo não será editado manualmente pelo usuário, sendo
  gerado automaticamente ao criar uma entrada na tabela utilizando a
  interface gráfica do Django Admin. Seu propósito é armazenar e entregar os
  parâmetros de filtragem definidos pela coluna keyword.
\end{description}

\subsection{Templates}

Os templates das páginas utilizadas pelos subsites seguem a mesma 
  \textcolor{blue}
  {\href{https://github.com/geosolutions-it/geonode-subsites/tree/main?tab=readme-ov-file#folder-structure}
  {organização}} 
utilizada pelo geonode-subsites. Sendo possível personalizar a
homepage através do arquivo "templates/index.html", a página de categorias
através do arquivo "templates/subsites/common/categories.html" e páginas
específicas de cada subsite através de arquivos na pasta
"templates/subsites/\textit{nome\_do\_subsite}".

A página inicial original do Geonode foi modificada para mostrar cartões de
cada subsite registrado no aplicativo Home, esta página herda suas
características da página original do Geonode então as funcionalidades de
customização de tema também influenciam esta página
(\textcolor{red}{\emph{CONFIRMAR}}). A página inicial de cada subsite foi
modificada para mostrar cartões de cada categoria registrada no aplicativo
Home, está página também herda da homepage original do Geonode e está sujeita
às mesmas customizações.

\section{Guia de cadastro de subsites e categorias}

Nesta seção do texto será exposto um breve processo para a adição de um novo
subsite e uma nova categoria através da interface gráfica de administrador do
Django. 

Para realizar o processo, o usuário deve ter acesso a uma credencial com
permissões de administração (não necessariamente super usuário). Uma vez com a
credencial correta, a página de administração pode ser acessada clicando no
ícone do perfil do usuário e no submenu Admin. Os cadastros serão feitos sob os
membros do grupo do aplicativo Home: Extended Subsites e Categories.

\section{Extended Subsites}

Na página de administração dos Extended Subsites é possível visualizar, editar,
criar e excluir os subsites cadastrados.



\end{document}
