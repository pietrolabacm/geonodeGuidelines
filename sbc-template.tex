\documentclass[12pt]{article}

\usepackage{sbc-template}
\usepackage{graphicx,url}
\usepackage[utf8]{inputenc}
%\usepackage[brazil]{babel}
%\usepackage[latin1]{inputenc}  

     
\sloppy

\title{Guia para o uso do Geonode}

\author{NDS\inst{1}}


\address{
  NDS
\nextinstitute
  CPRM
\email{\{cprm,nds\}@inf.ufrgs.br}
}

\begin{document} 

\maketitle


\section{Introdução}

Este documento tem como objetivo definir procedimentos a serem utilizados para
a carga de arquivos e metadados no ambiente Geonode do SGB.

\section{Acesso} \label{sec:firstpage}

Para carregar arquivos no sistema Geonode é necessário que o usuário esteja
conectado em uma conta com permissão para a carga. Para isso o usuário deve
seguir o link na parte superior direita da página para a tela de login e
inserir suas credenciais (ref{fig:login}).

\section{Dados}

Com o acesso ao usuário com permissões para carga será possível visualizar o
menu de "Dados" no canto superior esquerda da página. Duas opções estão
disponíveis: datasets e documentos. Datasets se referem aos arquivos de cunho
de representação geoespacial (shapefiles, geopackages, geojson, geotiff e
outros) e documentos se referem aos demais arquivos de registro de dados (pdf,
jpeg, log, csv, zip e outros).

Após acessar o menu Dados-Datasets ou Dados-Documentos será possível carregar
novos dados utilizando o botão "Adicionar Recurso", no canto superior direito
da página. Em uma nova página o usuário poderá utilizar o recurso de arrastar e
soltar ou o botão de selecionar arquivos no canto esquerdo da página, é
possível carregar mais de um arquivo durante este processo. Com os arquivos
devidamente selecionados o botão de "Upload" estará disponível para prosseguir
o carregamento. 

\textbf{Atenção, no caso de shapefiles será necessário que todos os seus
componentes sejam selecionados ou que todos os componentes sejam comprimidos em
um arquivo zip.}

Com a conclusão do upload, o usuário será redirecionado para uma página
listando todos os arquivos carregados, cada um apresentando um botão de
exclusão no caso de algum erro. Ao clicar no nome do arquivo o usuário poderá
acessar sua página de detalhamento, o mesmo pode ser feito através do menu
"Dados" acessado anteriormente.

Na página de detalhamento do dado é possível visualizar, filtrar e editar os
dados geoespaciais.

\section{Metadados}





\bibliographystyle{sbc}
\bibliography{sbc-template}

\end{document}
